\newpage
\section{NETCONF - Network Configuration Protocol}
\label{Section:NETCONF}

The Network Configuration Protocol (NETCONF) was first published in December 2006 as RFC 4741\cite{RFC:RFC4741:2006}. It was created as the successor to SNMP (see Section \ref{Section:SNMP}) to modernize, simplify, and unify network management.  The protocol was updated in June 2011 with RFC 6241\cite{RFC:RFC6241:2011}. The updated version defines the protocol as follows:

NETCONF uses remote procedure calls (RPC) to send and receive data. The data are encoded in XML. For communication between client and server, secure and connection-oriented sessions are used. Administrative devices are called clients, while network devices are called servers. A server can support one or more sessions. Configuration changes can only be done by authorized sessions and the effects of these changes must be visible to all active sessions.

It is possible for the client to discover the capabilities of a server. This helps the client to adapt its behavior to the capabilities of the server. Capabilities can have dependencies on other capabilities and a server must support all depending capabilities to support a capability.

The protocol is separated into four layers:

\begin{minipage}{\textwidth}
\begin{itemize}
    \item Secure Transport (Layer 1) -  The task of this layer is to provide a path between client and server across the network. The path can be provided via any protocol that fulfills the basic requirements.
    \item Messages (Layer 2) - This layer provides a mechanism for encoding RPCs and notifications.
    \item Operations (Layer 3) - Contains a set of base protocol operations.
    \item Content (Layer 4) - Contains data of the device.
\end{itemize}
\end{minipage}

To specify data models, the YANG data model language(\cite{RFC:RFC6020:2010}) was created. It covers the operations and content layer.

Transport protocols need to provide reliable and sequenced data delivery connections and automatically release requested resources on connection close. Moreover, authentication, data integrity, confidentiality, and replay protection must be provided. The protocol is responsible for establishing the connection between server and client. A NETCONF peer assumes that the authentication is handled by the protocol. Every NETCONF implementation must provide and support the SSH mapping (\cite{RFC:RFC6242:2011}).