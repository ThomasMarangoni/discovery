\section{Issues}
\label{Section:CAF-Issues}

If two devices from the same model and vendor have different firmware versions, it can happen that the devices return different types of data at the same MIB entry. To avoid this problem, the devices should have the same firmware and package versions.

Multiple OIDs can point to the same MIB entry. It is common that every manufacturer has its MIB table beside the standardized ones from IEEE. This makes it hard to find the needed MIB entries. A common way to find the needed entry is to perform an SNMP walk of the complete MIB table of the device. If the values of the entries are known, the entries can be found that way. The project Net-SNMP provides an application \textit{snmptranslate} to convert between numbered and labeled OIDs. This translation process can also help to find the correct entry.

By default, MIB entries can have different access permissions, depending on the configuration of the device. A recommended way to fix such issues is to create an \textit{SNMP view} on each device. The view defines which community can view what MIB entries.

Because manufacturers implement MIB tables differently, on some devices it was possible to get the MAC address of the remote port. But it was not possible to identify which local port received this information. All received information was assigned to local port zero, which is representing the bridge and not a specific port.

Some MIB entries of some manufacturers included non-printable characters, which represent different control characters in ASCII. Especially the entry \textit{SNMPv2-MIB::sysDescr} included them, which caused errors when parsing the output of \textit{snmpwalk}.

The application needs SNMPv1 for the deep scan and traps and needs SNMPv2c for scanning devices. Both protocols need to be activated. On some devices, it is required to activate the protocol version separately. Others activate all versions if SNMP is enabled.

To match the connection between ports, each port needs to have an individual MAC address. But some devices only provide one MAC address for all ports.

One device returned a MAC address for a remote port, that could not be found on the device, neither in the command line interface nor via SNMP. With this behavior, it was not possible to match the local and remote ports.

There is no default way to get the length of a connection between two devices. The reason for this is that there is no application in normal Ethernet networks that would require the length of the line. It is possible to get the length of a cable by using the Precision Time Protocol (PTP). It is used to precisely synchronize time between multiple network devices. If a precision time measurement is available between devices, it can be used to measure how long a message took from one device to another. Because the speed of electrical signals in different materials is known, the distance the signal traveled can be calculated. But while researching implementations of it, no daemon has been found that hand this data to the SNMP daemon.

To calculate better routes, the material of the transmission medium is also needed. The reason for that is the difference in latency from the material that is used for a connection. The most common materials are fiber optical and twisted-pair copper for wired connections. There is no standardized way to get this information from the devices. This is due to that the devices often do not have this information. There are MIB tables to get this information from some devices, but they are proprietary and only for specific devices.

To update information on changes on the network, the application is relying on SNMPv1 traps. All nodes of the network that are able to have more than one connection need to send them to the application. In the current state of the application, these traps need to be configured manually. Every device manufacturer has other methods and schemas to configure them and they are often also not well documented. 