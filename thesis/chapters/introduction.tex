Industrial systems have been changing for years to become more efficient and dynamic. With the occurrence of computers and Programmable Logic Controllers (PLCs), the era of the third industrial revolution was started. The production lines got more automated and more computers got integrated. The sensors and actuators are connected over different bus systems with a computing unit. This unit is then connected over Ethernet to other units. The network then can be connected to a management system.

With the fourth industrial revolution, which is happening in the last years, even more, automated processes got added. The bus systems got reduced and the sensors got integrated into the network of the computing units. Now the network reaches from the sensors and actuators to the ERP systems. There has also been a shift from proprietary network systems to systems that resemble Ethernet. These networks often also have support for the Internet Protocol version 4 and version 6 (IPv4 and IPv6). Because of the widespread availability of Ethernet and IP networks, the industrial implementations of them are easy to understand for the technician and also easy to implement for the manufacturers.

In order to attain its goals, Industry 4.0 systems should be interoperable, easily virtualized, decentralized, be capable of real-time processing, be service-oriented and be appropriately modular for increased reconfigurability and reusability. An open and vendor-independent network standard for real-time communication is Time-Sensitive Networking (TSN). It works similar to standard Ethernet but needs special switches and routers to function. In a TSN network, routes get calculated to guarantee timings and bandwidths. There are centralized and decentralized approaches to calculate these routes. The centralized approach is simpler and easier to implement, but if the central component fails routes can't be calculated anymore. An important element of complex network infrastructures is the automatic discovery of its parts, such as devices and the links in between, because it is the basis for rapid reconfiguration. \cite{IEEE:IND4:2019}

\section{Motivation}
Modern industrial assembly lines need to be able, to change their manufacturing focus as fast as possible. The time that is needed to change the production lines defines how fast the production can react to sudden events or new requirements. Open and vendor-independent protocols can help improve the independence of single hardware manufacturers.

TSN needs to be aware of the network topology to calculate network routes. In current implementations, this information must be manually added. This can be very time intensive depending on the size of the network and the changes. An automatic way to collect all needed information can reduce the time needed for changes drastically. It also allows to detect changes while the production is running and react to them. This makes it possible to calculate and apply new network routes on the fly. To increase the performance of searching better network routes, additional parameters are needed. This thesis assumes the following parameters to be helpful for such a task: the length of a connection, the medium of a connection, and the link speed. These few parameters can already improve the pathfinding on the network by a lot, because the limits of every connection are known exactly.

By discovering the direct neighbors from each device, the topology of the network can be generated. For discovering the direct neighbors, the vendor-independent and open Link Layer Discovery Protocol (LLDP) is utilized. To collect the connection and additional information, the also vendor-independent and open Simple Network Manager Protocol (SNMP) is used. To allow changing the production line while in operation, the system must be aware of changes on the network and they should be detected in under one second. Short detection times allow faster reconfigurations of the network, which reduces the downtime of the production. SNMP can detect changes on network interfaces of devices and send a notification to a specific host.

\newpage
\section{Structure of the work}
The remaining chapters are structured in the following.
\begin{itemize}
    \item \textbf{Chapter \ref{Chapter:stateoftheart} - State of the Art:} This chapter gives an introduction about the technologies and standards used. It covers MIB, SNMP, LLDP and NETCONF.
    \item \textbf{Chapter \ref{Chapter:concept} - Concept:} It describes the theoretical approach of the application to get the data from the network.
    \item \textbf{Chapter \ref{Chapter:implementation} -  Implementation:} This chapter describes the practical approach of the implementation and its structure.
    \item \textbf{Chapter \ref{Chapter:evaluation} - Evaluation:} This chapters describes how the application was tested, how to setup a test environment and what requirements must be met by the network devices.
    \item \textbf{Chapter \ref{Chapter:conclusionandfuturework} - Conclusion and Future Work:} It describes what issues have been found, the conclusion from them and possible improvements that can be done in future implementations.
\end{itemize}