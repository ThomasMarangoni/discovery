\section{Improvements}
\label{Section:CAF-Improvements}

It is possible to improve the speed of the initial deep scan, by reducing the interval packages are sent. This would require a network with bigger bandwidth capabilities and also a device that can handle this amount of traffic. Another way would be increasing the interval, which would reduce the bandwidth needed and allow to increase the number of devices that can be scanned. It would take longer to scan the whole network, but it enables to cover larger networks.

The current implementation uses snmpwalk as a method to query the needed data. Each needed MIB table is requested by a snmpwalk, to reduce the time needed for the operation. There is an advanced implementation called \textit{snmpbulkwalk}, which uses the SNMP BULKWALK request to gather the data. It is faster by bundling the requested data in one request. With \textit{snmpbulkwalk}, it may be faster to request a higher OID where all needed MIB tables are included. The speed improvement depends on the OID of these MIB tables and their sizes.

Instead of using the precompiled \textit{snmpwalk} executable, it would be better to use the SNMP library provided by the net-snmp project. It makes the requests more flexible and could make it possible to parallelize the snmpbulkwalk. This may increase the speed of the initialization phase of the application significantly.

The communication to SNMP devices is not secure as SNMPv2c is used for the application. In some networks, it could be a requirement to secure the connections. This is possible with SNMPv3. SNMPv3 also allows to set different permissions per user. In combination with authentication, this could restrict the publicly available information. The last possible security measurement is transport security, which allows to create a secure connection or session between two points. However, this security measurement may introduce a new complexity that makes it harder to maintain the network.

IPv6, the successor of IPv4, is getting more popular, the current state of the application does not support IPv6. With IPv6 it would be possible to assign one IPv6 address per port. Then it is possible to use the IPv6 Neighbor Discovery Protocol \cite{RFC:RFC4861:2007}. This protocol is similar to LLDP (see Section \ref{Section:LLDP}) and may be able to replace it.

Because SNMP was designed in 1988 and uses MIB for data representation, it does not follow modern implementation models. Fields that can have data represented in multiple ways need a separate field that saves the representation form. This makes it harder to parse data queried by SNMP. NETCONF, the successor of SNMP, uses YANG as a data modeling language. In YANG, data is represented as objects, whereby each property of an object has a fixed data type assigned. Another advantage is, interrelations between different data points can be represented. Using NETCONF and YANG would reduce the work needed to implement data parsing and interoperability between programs, because not every MIB entry has to be linked manually. YANG models can be shared between applications with JSON or XML. Both file formats are widely used for transferring information.
